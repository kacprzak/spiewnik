\beginsong{Катюша}[cr={1938},by={M. Isakovsky, M. Blanter}]
  \beginverse
    \[A\mi]{Расцветали} \[E7]{яблони} и груши,
    \[E7]{Поплыли} \[A\mi]{туманы} над рекой.
    \[A\mi]{Вы}\[F]{хо}\[C]{дила} \[D\mi]{на} берег \[A\mi]{Катюша},
    \[D\mi]{На} \[A\mi]{высокий} \[E7]{берег} на \[A\mi]{крутой}.
  \endverse
  \beginverse
    Выходила, песню заводила
    Про степного, сизого орла,
    Про того, которого любила,
    Про того, чьи письма берегла.
  \endverse
  \beginverse
    Ой ты, песня, песенка девичья,
    Ты лети за ясным солнцем вслед.
    И бойцу на дальнем пограничье
    От Катюши передай привет.
  \endverse
  \beginverse
    Пусть он вспомнит девушку простую,
    Пусть услышит, как она поет,
    Пусть он землю бережет родную,
    А любовь Катюша сбережет.
  \endverse
  \beginverse
    Расцветали яблони и груши,
    Поплыли туманы над рекой.
    Выходила на берег Катюша,
    На высокий берег на крутой.
  \endverse
\endsong
