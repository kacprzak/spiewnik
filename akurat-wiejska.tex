\begin{song}{Wiej-ska}{}{}{Akurat}{}{}
  \begin{SBVerse}
    \Ch{Hmin}{Gdy interes masz do posła} nie \Ch{D}{potraktuj} \Ch{A}{go} jak \Ch{Hmin}{osła}

    Tego z bajki, w której z głodu, odszedł osioł już za młodu

    Osiołkowi w żłoby dano, w jednym owies, w drugim siano

    Pośród żłobów z których jada, z niezdecydowania pada
  \end{SBVerse}
  \begin{SBChorus}
    \Ch{Hmin}{Posły} os\Ch{F#}{ły} \Ch{D}{senato}\Ch{A}{ry}

    \Ch{D}{a na} to\Ch{A}{rach} \Ch{Hmin}{tłok} był spo\Ch{F#}{ry}

    Pobrudziły se kopyta

    stojąc w tłoku do koryta
  \end{SBChorus}
  \begin{SBVerse}
    Jak ten osioł, tak i poseł, może długo kręcić nosem

    Chyba żeby w jednym z koryt, poseł ujrzał spraw koloryt

    Gdy zrozumie czym to pachnie, poseł może wybrać trafnie

    Gdy koryto ładnie przybrać, poseł może dobrze wybrać
  \end{SBVerse}
  \begin{SBChorus}
    Posły osły senatory...
  \end{SBChorus}
  \begin{SBVerse}
    Chociaż nosi garnitury, poseł knąbrny jest z natury

    Poseł wierzy w słuszność sprawy, gdy załącznik jest ciekawy

    Więc pamiętaj aby zgodnie żyć z posłami, trzeba gdzieś między wierszami

    Zawrzeć swojej sprawy sedno, wszak im nie jest wszystko jedno
  \end{SBVerse}
  \begin{SBChorus}
    Posły osły senatory...
  \end{SBChorus}
\end{song}
