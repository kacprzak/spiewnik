\sclearpage
\beginsong{Jesteśmy na Wczasach}[by={Wojciech Młynarski}]
  \beginverse
    {\nolyrics Intro: \[C#\m] \[F#\m] \[C#\diminished] \[G#]}

    \[C#\m]{Za oknami noc}, w górach śniegu moc okrywa \[D#7]{wszystko.}
    \[F#\m6]{Czort jedyny wie}, \[G#]{co rzuciło mnie w to} \[C#\m]{uzdrowisko.}
    \[G#\halfdim]{Na parkiecie szum}, \[C#]{wczasowiczów tłum spleciony} \[F#\m]{gęsto.}
    \[D#7]{Siedzę tutaj sam}, a przed sobą mam orkiestrę \[G#]{męską.}
  \endverse
  \beginverse
    Typ, co szarpie bas, wie, że nadszedł czas, gdy w kimś na bańce
    Czuła struna drgnie i rozpoczną się góralskie tańce.
    Jest górala wart, taniec gdy masz fart, gdy dziewczę kwili.
    \[D#7]{Z basem typ to wie}, więc szykuje się i \[G#]{już po} chwili: \[B7]{}
  \endverse
  \beginverse
    Dla sympatycznej panny Krysi z turnusu trzeciego,
    od sympatycznego pana Waldka: pucio-pucio!
  \endverse
  \beginchorus
    \[E]{Jesteśmy na wczasach, w tych góralskich lasach,}
    W promieniach \[F#\m]{słonecznych} \[B]{} opalamy \[E6]{się.}
    Orkiestra przygrywa skocznego begina,
    To nie twoja wina, że podrywam cię...
  \endchorus
  \beginverse
    \[C#\m#9]{Ta panna Krysia, panna Krysia}
    \[C#\m]{Królowała na turnusach nie od dzisiaj,}
    \[D#7]{A każdego roku, właśnie o tej porze}
    \[A#\dimseven]{Przyjeżdżała tu do pensjonatu} \[G#]{``Orzeł''.}
    \[C#\m9]{Kuracjuszy rozmarzony wzrok}
    \[A7]{Śledził wciąż jej każdy gest} i \[G#]{krok...}
  \endverse
  \beginverse
    Za oknami noc, w górach śniegu moc na drzewach wisi,
    czort jedyny wie, że basista też się kocha w Krysi...
    Wie jedyny czort, co kosztuje to, by wciąż od nowa
    brać kontrabas i tłumiąc pożar krwi tak anonsować:
  \endverse
  \beginverse
    Dla sympatycznej panny Krysi z turnusu trzeciego
    od sympatycznego oczywiście niewątpliwie pana Mietka: pucio-pucio!
  \endverse
  \beginchorus
    Jesteśmy na wczasach w tych góralskich lasach,
    w promieniach słonecznych opalamy się.
  \endchorus
  \beginverse
    A panna Krysia, panna Krysia
    z panem Mietkiem, co się tuż przed chwilą przysiadł
    przemierzała wzdłuż i wszerz parkietu przestrzeń,
    ale nigdy nie spojrzała ku orkiestrze,
    skąd basisty rozmarzony wzrok
    śledził wciąż jej każdy gest i krok.
  \endverse
  \beginchorus
    Za oknami noc, w górach śniegu moc okrywa wszystko,
    cały turnus śpi, a wśród innych śni i nasz basista,
    że dokoła szum, na parkiecie tłum, przy czołach czoła,
    a on rzuca bas i ma w oczach blask i głośno woła:
  \endchorus
  \beginverse
    ``Spokój Orkiestra!!!
    Teraz... dla sympatycznej panny Krysi...
    z turnusu trzeciego... ode mnie...
    Panno Krysiu... kocham panią!... Wszystko...''
  \endverse
  \beginverse
    Ha ha ha ha ha ha ha ha!!! Co to się działo, co się działo!
    Uzdrowiska pół ze śmiechu sie skręcało
    i skręciło by do końca biednych ludzi,
    gdyby wreszcie się basista nie obudził...
    \[E\majseven]{Bo miewamy często} \[F#\m7]{głupie} \[G#\m7]{sny,} \[F#\m7]{}
    \[E\majseven]{ale potem się} \[F#\m7]{budzimy} \[G#]{i:}
  \endverse
  \beginverse
    Dla sympatycznej panny Krysi z turnusu trzeciego
    od sympatycznego, niewątpliwie, pana Waldka: pucio-pucio!
  \endverse
\endsong
