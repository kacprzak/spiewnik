\beginsong{Bieszczadzkie Anioły}[by={Stare Dobre Małżeństwo}]
\beginverse
\[A\mi]Anioły są takie ciche
\[G]Zwłaszcza te w Bieszczadach
\[A\mi]Gdy spotkasz takiego w górach
\[E\mi]Wiele z nim nie pogadasz
\endverse
\beginverse
\[C]Najwyżej na ucho ci \[G]powie
\[C]Gdy będzie w dobrym \[F]humorze
Że \[C]skrzydła nosi w \[G]plecaku
\[A\mi]Nawet \[E\mi]przy dobrej po\[A\mi]godzie
\endverse
\beginverse
Anioły są całe zielone
Zwłaszcza te w Bieszczadach
Łatwo w trawie się kryją
I w opuszczonych sadach
\endverse
\beginverse
W zielone grają ukradkiem
Nawet karty mają zielone
Zielone mają pojęcie
A nawet zielony kielonek
\endverse
\beginchorus
\[C]Anioły biesz\[G]czadzkie, bieszczadzkie \[A\mi]anioły
Dużo w was radości i dobrej pogody
Bieszczadzkie anioły, anioły bieszczadzkie
Gdy skrzydłem cię trącą już jesteś ich bratem
\endchorus
\beginverse
Anioły są całkiem samotne
Zwłaszcza te w Bieszczadach
W kapliczkach zimą drzemią
Choć może im nie wypada
\endverse
\beginverse
Czasem taki anioł samotny
Zapomni dokąd ma lecieć
I wtedy całe Bieszczady
Mają szaloną uciechę
\endverse
\beginchorus
Anioły bieszczadzkie, bieszczadzkie anioły ...
\endchorus
\beginverse
Anioły są wiecznie ulotne
Zwłaszcza te w Bieszczadach
Nas też czasami nosi
Po ich anielskich śladach
\endverse
\beginverse
One nam przyzwalają
I skrzydłem wskazują drogę
I wtedy w nas się zapala
Wieczny bieszczadzki ogień
\endverse
\beginchorus
Anioły bieszczadzkie, bieszczadzkie anioły ... x3
\endchorus
\endsong
