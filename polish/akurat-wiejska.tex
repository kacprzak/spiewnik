\beginsong{Wiej-ska}[by={Akurat}]
  \beginverse
    \[B\m]{Gdy interes masz do posła} nie \[D]{potraktuj} \[A]{go} jak \[B\m]{osła}
    Tego z bajki, w której z głodu, odszedł osioł już za młodu
    Osiołkowi w żłoby dano, w jednym owies, w drugim siano
    Pośród żłobów z których jada, z niezdecydowania pada
  \endverse
  \beginchorus
    \[B\m]{Posły} os\[F#]{ły} \[D]{senato}\[A]{ry}
    \[D]{a na} to\[A]{rach} \[B\m]{tłok} był spo\[F#]{ry}
    Pobrudziły se kopyta
    stojąc w tłoku do koryta
  \endchorus
  \beginverse
    Jak ten osioł, tak i poseł, może długo kręcić nosem
    Chyba żeby w jednym z koryt, poseł ujrzał spraw koloryt
    Gdy zrozumie czym to pachnie, poseł może wybrać trafnie
    Gdy koryto ładnie przybrać, poseł może dobrze wybrać
  \endverse
  \beginchorus
    Posły osły senatory...
  \endchorus
  \beginverse
    Chociaż nosi garnitury, poseł knąbrny jest z natury
    Poseł wierzy w słuszność sprawy, gdy załącznik jest ciekawy
    Więc pamiętaj aby zgodnie żyć z posłami, trzeba gdzieś między wierszami
    Zawrzeć swojej sprawy sedno, wszak im nie jest wszystko jedno
  \endverse
  \beginchorus
    Posły osły senatory...
  \endchorus
\endsong
