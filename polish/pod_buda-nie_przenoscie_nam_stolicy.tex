\beginsong{Nie Przenoście Nam Stolicy do Krakowa}[by={Pod Budą}]
\beginverse
{\nolyrics Intro: \[G] \[A] \[D] \[D] | \[G] \[A] \[D] \[D]}
\endverse
\beginverse
Złote \[D]nuty spadają na Rynek
I mu\[G]zyki dokoła jest w bród
Po kró\[E\mi]lewsku gotuje Wierzynek
A kwia\[D]ciarki cze\[G]kają na \[A]cud
\endverse
\beginverse
Czasem we śnie pojawi się poseł
który rację ma zawsze i basta
I uczonym oznajmia mi głosem
że najlepiej nam było za Piasta
\endverse
\beginverse
\[G]Wielkie nie\[D]ba \[A]co ja \[D]słyszę
\[G]Wielkie nie\[A]ba co się \[D]śni
Wstaję \[G]rano prędko \[D]piszę
Krótki \[E\mi]refren zdania \[A]trzy
\endverse
\beginchorus
Nie prze\[E\mi]noście nam sto\[A]licy do Kra\[D]kowa
Chociaż tak lubicie wracać do symboli
Bo się \[G]zaraz tutaj \[B7]zjawią
Buntne \[E\mi]miny, święte \[G\mi]słowa
I głu\[D]pota, która \[A]aż naprawdę \[D]boli
\endchorus
\beginchorus
U nas chodzi się z księżycem w butonierce
U nas wiosną wiersze rodzą się najlepsze
I odmiennym jakby rytmem
U nas ludziom bije serce
Choć dla serca nieszczególne tu powietrze
\endchorus
\beginverse
Złote nuty spadają na Rynek
I muzyki dokoła jest w bród
Po królewsku gotuje Wierzynek
A kwiaciarki czekają na cud
\endverse
\beginverse
Zasłuchani w historii kawałek
Który matka czytała co wieczór
Przeżywaliśmy bitwy wspaniałe
Nadążając jak zwykle z odsieczą
\endverse
\beginverse
To się jednak już zdarzyło
Deszcz nie jeden na nas spadł
Nie powtórzy się, co było
Inny dziś w kominach wiatr
\endverse
\beginchorus
Nie przenoście nam stolicy do Krakowa ...
\endchorus
\beginchorus
Nie przenoście nam stolicy do Krakowa
Niech już raczej pozostanie tam gdzie jest
\lrep Najgoręcej o to proszą
Dobrze ważąc własne słowa
Dwa Krakusy Grzegorz T i Andrzej S.\rrep \rep{2}
\endchorus
\endsong
