\beginsong{10 w skali Beauforta}[by={Krzysztof Klenczon}]
\beginverse
\[A\mi]Kołysał nas \[D\mi]zachodni wiatr,
\[E7]Brzeg gdzieś za rufą \[A\mi]został.
I \[D\mi]nagle ktoś jak \[A\mi]papier zbladł:
Sztorm \[B7]idzie, panie \[E7]bosman!
\endverse
\beginchorus
A \[F]bos\[C]man tylko \[F]zapiął \[C]płaszcz
I \[F]zaklął: - \[E7]Ech, do \[A\mi]czorta!
Nie \[F]daję \[G]łajbie \[A\mi]ża\[E7]dnych \[A\mi]szans!
\[F]Dziesięć w \[E7]skali Beau\[A\mi]forta!
\endchorus
\beginverse
Z zasłony ołowianych chmur
Ulewa spadła nagle.
Rzucało nami w górę, w dół,
I fala zmyła żagle.
\endverse
\beginchorus
A bosman \dots
\endchorus
\beginverse
Gdzie został ciepły, cichy kąt
I brzegu kształt znajomy?
Zasnuły mgły daleki ląd
Dokładnie, z każdej strony.
\endverse
\beginchorus
A bosman \dots
\endchorus
\beginverse
O pokład znów uderzył deszcz
I padał już do rana.
Piekielnie ciężki to był rejs,
Szczególnie dla bosmana.
\endverse
\beginchorus
A bosman tylko zapiął płaszcz
I zaklął: - Ech, do czorta!
Przedziwne czasem sny się ma!
Dziesięć w skali Beauforta! \rep{3}
\endchorus
\endsong
