\beginsong{Keja}[cr={1971},by={Jerzy Porębski}]
  \beginverse
    \[A\m]{Gdyby} ktoś przyszedł i powiedział - \[G]{stary} czy masz \[A\m]{czas?}
    \[C]{Potrzebuje} do załogi \[G]{jakąś} nowa \[C]{twarz}
    \[C7]{Amazonka}, Wielka \[F]{rafa}, oceany \[D\m]{trzy}
    \[A\m7]{Rejs} na całość: rok, \[E]{dwa} lata - \[A\m]{odpowiedziałbym}:
  \endverse
  \beginchorus
    \[A\m]{Gdzie} ta keja, a \[E7]{przy} niej ten \[A\m]{jacht}
    \[C]{Gdzie} ta \[G]{koja} wymarzona w \[C]{snach} \[C7]{}
    \[G\m]{Gdzie} te wszystkie \[A7]{sznurki} \[D\m]{od} \[A7]{tych} \[D\m]{szmat}
    \[A\m]{Gdzie} ta brama \[E7]{na szeroki} \[A\m]{świat}.
  \endchorus
  \beginchorus
    Gdzie ta keja, a przy niej ten jacht
    Gdzie ta koja wymarzona w snach
    W każdej chwili płynę w taki rejs
    Tylko gdzie to jest, gdzie to jest?
  \endchorus
  \beginverse
    Gdzieś na dnie wielkiej szafy leży ostry nóż
    Stare dżinsy wystrzępione impregnuje kurz
    W kompasie igła zardzewiała, lecz kierunek znam
    Biorę wór na plecy i przed siebie gnam.
  \endverse
  \beginchorus
    Gdzie ta keja, a przy niej ten jacht \dots
  \endchorus
  \beginverse
    Przeszły lata zapyziałe, rzęsą zarósł staw
    A na przystani czółno stało - kolorowy paw
    Zaokrągliły się marzenia, wyjałowiał step
    Lecz dalej marzy o załodze ten samotny łeb.
  \endverse
  \beginchorus
    Gdzie ta keja, a przy niej ten jacht \dots
  \endchorus
\endsong
