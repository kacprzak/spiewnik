\begin{song}{Keja}{A-moll}{1971}{Jerzy Porębski}{}{}
	\begin{SBVerse}
	\Ch{Am}{Gdyby} ktoś przyszedł i powiedział - \Ch{G}{stary} czy masz \Ch{Am}{czas}?

	\Ch{C}{Potrzebuje} do załogi \Ch{G}{jakąś} nowa \Ch{C}{twarz}

	\Ch{C7}{Amazonka}, Wielka \Ch{F}{rafa}, oceany \Ch{Dm}{trzy}

	\Ch{Am7}{Rejs} na całość: rok, \Ch{E}{dwa} lata - \Ch{Am}{odpowiedziałbym}:
	\end{SBVerse}

	\begin{SBChorus}
	\Ch{Am}{Gdzie} ta keja, a \Ch{E7}{przy} niej ten \Ch{Am}{jacht}

	\Ch{C}{Gdzie} ta \Ch{G}{koja} wymarzona w \Ch{C}{snach} \Ch{C7}{}

	\Ch{Gm}{Gdzie} te wszystkie \Ch{A7}{sznurki} \Chr{Dm}{od} \Ch{A7}{tych} \Ch{Dm}{szmat}

	\Ch{Am}{Gdzie} ta brama \Ch{E7}{na szeroki} \Ch{Am}{świat}.
	\end{SBChorus}

	\begin{SBChorus}
	Gdzie ta keja, a przy niej ten jacht

	Gdzie ta koja wymarzona w snach

	W każdej chwili płynę w taki rejs

	Tylko gdzie to jest, gdzie to jest?

	\end{SBChorus}

	\begin{SBVerse}
	Gdzieś na dnie wielkiej szafy leży ostry nóż

	Stare dżinsy wystrzępione impregnuje kurz

	W kompasie igła zardzewiała, lecz kierunek znam

	Biorę wór na plecy i przed siebie gnam.
	\end{SBVerse}

	\begin{SBChorus}
	Gdzie ta keja, a przy niej ten jacht...
	\end{SBChorus}

	\begin{SBVerse}
	Przeszły lata zapyziałe, rzęsą zarósł staw

	A na przystani czółno stało - kolorowy paw

	Zaokrągliły się marzenia, wyjałowiał step

	Lecz dalej marzy o załodze ten samotny łeb.
	\end{SBVerse}

	\begin{SBChorus}
	Gdzie ta keja, a przy niej ten jacht...
	\end{SBChorus}
\end{song}
