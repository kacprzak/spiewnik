\beginsong{Keja}{A-moll}{1971}{Jerzy Porębski}{}{}
  \beginverse
    \[Am]{Gdyby} ktoś przyszedł i powiedział - \[G]{stary} czy masz \[Am]{czas}?

    \[C]{Potrzebuje} do załogi \[G]{jakąś} nowa \[C]{twarz}

    \[C7]{Amazonka}, Wielka \[F]{rafa}, oceany \[Dm]{trzy}

    \[Am7]{Rejs} na całość: rok, \[E]{dwa} lata - \[Am]{odpowiedziałbym}:
  \endverse

  \beginchorus
    \[Am]{Gdzie} ta keja, a \[E7]{przy} niej ten \[Am]{jacht}

    \[C]{Gdzie} ta \[G]{koja} wymarzona w \[C]{snach} \[C7]{}

    \[Gm]{Gdzie} te wszystkie \[A7]{sznurki} \[Dm]{od} \[A7]{tych} \[Dm]{szmat}

    \[Am]{Gdzie} ta brama \[E7]{na szeroki} \[Am]{świat}.
  \endchorus

  \beginchorus
    Gdzie ta keja, a przy niej ten jacht

    Gdzie ta koja wymarzona w snach

    W każdej chwili płynę w taki rejs

    Tylko gdzie to jest, gdzie to jest?

  \endchorus

  \beginverse
    Gdzieś na dnie wielkiej szafy leży ostry nóż

    Stare dżinsy wystrzępione impregnuje kurz

    W kompasie igła zardzewiała, lecz kierunek znam

    Biorę wór na plecy i przed siebie gnam.
  \endverse

  \beginchorus
    Gdzie ta keja, a przy niej ten jacht...
  \endchorus

  \beginverse
    Przeszły lata zapyziałe, rzęsą zarósł staw

    A na przystani czółno stało - kolorowy paw

    Zaokrągliły się marzenia, wyjałowiał step

    Lecz dalej marzy o załodze ten samotny łeb.
  \endverse

  \beginchorus
    Gdzie ta keja, a przy niej ten jacht...
  \endchorus
\endsong
