\beginsong{Czarny chleb i czarna kawa}{A-moll}{}{Jerzy Filas}{}{}
  \beginverse
    \[Am]{Jedzie} pociąg, złe wagony, do więzie\[C]{nia} wiozą mnie

    Świat ma \[G]{tylko} cztery strony a w tym \[Am]{świecie} nie ma mnie.
  \endverse
  \beginverse
    Gdy swe oczy otworzyłem, wielki żal ogarnął mnie

    Po policzkach łzy spłynęły, zrozumiałem wtedy, że
  \endverse
  \beginverse{2x}
    Czarny chleb i czarna kawa, opętani samotnością

    Myślą swą szukają szczęście, które zwie się wolnością
  \endverse
  \beginverse
    Młodsza siostra zapytała: mamo gdzie braciszek mój?

    Brat Twój w ciemnej celi siedzi, odsiaduje wyrok swój
  \endverse
  \beginchorus
    Czarny chleb i czarna kawa, opętani samotnością

    Myślą swą szukają szczęście, które zwie się wolnością
  \endchorus
  \beginverse
    Wtem do celi klawisz wpada i zaczyna więźnia bić

    Młody więzień na twarz pada, serce mu przestaje bić
  \endverse
  \beginverse
    I nadejdzie chwila błoga, śmierć zabierze oddech mój

    moje ciało stąd wyniosą, a pod celą będą znów:
  \endverse

  \beginverse{2x}
    Czarny chleb i czarna kawa, opętani samotnością

    Myślą swą szukają szczęście, które zwie się wolnością
  \endverse
\endsong
