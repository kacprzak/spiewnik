\beginsong{Muchy}{A-moll}{1984}{Nijak}{}{}

  \beginverse{Intro}
    Oj, \[Am]{było}, było \[E7]{lato}, oj, było, było \[Am]{lato}...

    Oj, było, było lato, oj, było, było lato...
  \endverse

  \beginverse
    Z mych wakacji to wspomnienia będą jednych,

    moich uwag zbiór świadczący o tym, że

    świat owadów może być nad wyraz wredny

    i w dodatku prześladuje ciągle mnie.
  \endverse
  \beginverse
    Będąc szczupłym jeszcze latem raz poznałem

    piękne dziewczę wykształcone tu i tam.

    Że bez mamy wypoczywa tak mniemałem,

    więc zdobycia jej uknułem chytry plan.
  \endverse
  \beginverse
    Przedstawiłem się nazajutrz jej na plaży,

    mówiąc: Proszę, tu na kocu miejsce mam.

    I choć słońce dwa pośladki moje smaży,

    w oczy patrzę jej, Don Chuana minę mam.
  \endverse
  \beginverse
    Kremem prosi, bym jej plecy wysmarował,

    więc przylgnąłem jak pijawka do jej ciała.

    Nagle mówi: Wie pan, żar mnie zmitygował,

    w chłodnym gaju bym pospacerować chciała.
  \endverse
  \beginverse
    Ach, bogowie! Czegóż więcej pragnąć mogłem,

    brzuch wciągnąłem, wyprężyłem nagi tors.

    I w najbardziej gęste chaszcze ją powiodłem,

    ręką obejmując piękny, jędrny gors.
  \endverse
  \beginverse
    Spacer zmęczył dziewczę pośród pni zwalonych,

    lekko wsparła głowę na ramieniu mym,

    Ja wargami szukam ust jej rozchylonych,

    już znalazłem, już chcę wpić się, gdy wtem...
  \endverse

  \beginchorus
    Pierdolone muchy i komary jebane, pierdolone muchy i komary jak, ptfuj!
  \endchorus

  \beginverse
    Co się stało pyta - boi się pan muszek?

    Nie odrzekłem mam nerwowy tik.

    Znów dziewiczej nieśmiałości lody kruszę,

    już puls szybciej bije jej, wtem nagle ...~bzyk.
  \endverse

  \beginchorus
    Pierdolone muchy i komary jebane, pierdolone muchy i komary jak chuj!
  \endchorus

  \beginverse
    Ciało w bąblach jest na plaży niezbyt piękne,

    zniechęcony całkiem do miłosnych uciech,

    Wdziałem koszulinę lekką i spodenki,

    zapragnąłem spić zimnego piwka kufel.
  \endverse
  \beginverse
    A, lecz wiadomo, lato, żar, skąd piwo zimne?

    Nie Miami Beach to nawet i nie Soczi.

    I choć Heineken mi się śnił, czy jakieś inne,

    to nawet w słupskim piwie chciałbym usta zmoczyć.
  \endverse
  \beginverse
    Poszukiwania prowadziłem przez dzień cały,

    już zrezygnować chciałem, miałem tego dość,

    Wtem widzę: facet się zatacza znietrzeźwiały,

    podchodzę, wącham piwem śmierdzi gość.
  \endverse
  \beginverse
    Ach, zbawco zakrzyknąłem - gdzieś się urżnął?!

    I chociaż biedak mówić już nie może,

    zachody moje były nie na próżno,

    bo wolno wybełkotał: Tam jest rożen.
  \endverse
  \beginverse
    I znalazłem w końcu pragnień moich przystań,

    tę oazę, gdzie wypoczywają męże.

    Gdzie kufelek piwka wzmacnia wódka czysta,

    co najlepszym w walce z kacem jest orężem.
  \endverse
  \beginverse
    Barman wprawną ręką kurek już odkręcił,

    szczerze wlewa mi spieniony złoty płyn.

    Tak, do piwa chyba nic mnie nie zniechęci,

    lecz rzuciłem okiem w kufel, a tu w nim...
  \endverse

  \beginchorus
    Pierdolone muchy i komary jebane, pierdolone muchy i komary jak, ptfuj!
  \endchorus

  \beginverse
    Komary jak \[Am]{Sodoma}, mucha jak \[Dm]{Gomora}!

    Pozostało mi jedyne wyjście \[Am]{uciec}.

    Tak skończyło się to lato na \[E7]{jeziorach},

    pozbawiły mnie owady wszelkich \[Am]{uciech}.
  \endverse

  \beginchorus
    Pierdolone muchy i komary jebane, pierdolone muchy i komary jak chuj!
  \endchorus

  \beginverse
    Bąble dawno już otęchły, minął niesmak,

    miejski zgiełk utulił dawno w sercu żal.

    Na wycieczkach w Gerlitz wprawiam się w niemieckim,

    wołam: Ober eine flaschen beer noch mal!
  \endverse
  \beginverse
    Lecz, gdy wracam z tej wycieczki, dajmy na to,

    różne myśli biją w mój gorący łeb.

    Czy mi kumpel nie pokaże coś pod klapą,

    czy za klapy mnie nie złapie jakiś cep?
  \endverse
  \beginverse
    Gdy tak sam się po mieszkaniu w nocy tłukę,

    nie ma kogoś, kto na ramię głowę złoży,

    Zatwardziałą duszę ściśnie nieraz smutek

    i jak gdyby w serce wbił ktoś ostre noże.
  \endverse
  \beginverse
    Wtedy komar, każda muszka jest mi druhem,

    za ich skrzydeł brzękiem dusza mi się rwie,

    do natury, starych lip owianych puchem,

    do tych pni zwalonych, krzaków, haszczy, gdzie...
  \endverse

  \beginchorus
    Pierdolone muchy i komary jebane, pierdolone muchy i komary jak chuj!
  \endchorus
\endsong
