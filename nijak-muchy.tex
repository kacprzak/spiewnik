\begin{song}{Muchy}{A-moll}{1984}{Nijak}{}{}

  \begin{SBBracket}{Intro}
    Oj, \Ch{Am}{było}, było \Ch{E7}{lato}, oj, było, było \Ch{Am}{lato}...

    Oj, było, było lato, oj, było, było lato...
  \end{SBBracket}

  \begin{SBVerse}
    Z mych wakacji to wspomnienia będą jednych,

    moich uwag zbiór świadczący o tym, że

    świat owadów może być nad wyraz wredny

    i w dodatku prześladuje ciągle mnie.
  \end{SBVerse}
  \begin{SBVerse}
    Będąc szczupłym jeszcze latem raz poznałem

    piękne dziewczę wykształcone tu i tam.

    Że bez mamy wypoczywa tak mniemałem,

    więc zdobycia jej uknułem chytry plan.
  \end{SBVerse}
  \begin{SBVerse}
    Przedstawiłem się nazajutrz jej na plaży,

    mówiąc: Proszę, tu na kocu miejsce mam.

    I choć słońce dwa pośladki moje smaży,

    w oczy patrzę jej, Don Chuana minę mam.
  \end{SBVerse}
  \begin{SBVerse}
    Kremem prosi, bym jej plecy wysmarował,

    więc przylgnąłem jak pijawka do jej ciała.

    Nagle mówi: Wie pan, żar mnie zmitygował,

    w chłodnym gaju bym pospacerować chciała.
  \end{SBVerse}
  \begin{SBVerse}
    Ach, bogowie! Czegóż więcej pragnąć mogłem,

    brzuch wciągnąłem, wyprężyłem nagi tors.

    I w najbardziej gęste chaszcze ją powiodłem,

    ręką obejmując piękny, jędrny gors.
  \end{SBVerse}
  \begin{SBVerse}
    Spacer zmęczył dziewczę pośród pni zwalonych,

    lekko wsparła głowę na ramieniu mym,

    Ja wargami szukam ust jej rozchylonych,

    już znalazłem, już chcę wpić się, gdy wtem...
  \end{SBVerse}

  \begin{SBChorus}
    Pierdolone muchy i komary jebane, pierdolone muchy i komary jak, ptfuj!
  \end{SBChorus}

  \begin{SBVerse}
    Co się stało pyta - boi się pan muszek?

    Nie odrzekłem mam nerwowy tik.

    Znów dziewiczej nieśmiałości lody kruszę,

    już puls szybciej bije jej, wtem nagle ...~bzyk.
  \end{SBVerse}

  \begin{SBChorus}
    Pierdolone muchy i komary jebane, pierdolone muchy i komary jak chuj!
  \end{SBChorus}

  \begin{SBVerse}
    Ciało w bąblach jest na plaży niezbyt piękne,

    zniechęcony całkiem do miłosnych uciech,

    Wdziałem koszulinę lekką i spodenki,

    zapragnąłem spić zimnego piwka kufel.
  \end{SBVerse}
  \begin{SBVerse}
    A, lecz wiadomo, lato, żar, skąd piwo zimne?

    Nie Miami Beach to nawet i nie Soczi.

    I choć Heineken mi się śnił, czy jakieś inne,

    to nawet w słupskim piwie chciałbym usta zmoczyć.
  \end{SBVerse}
  \begin{SBVerse}
    Poszukiwania prowadziłem przez dzień cały,

    już zrezygnować chciałem, miałem tego dość,

    Wtem widzę: facet się zatacza znietrzeźwiały,

    podchodzę, wącham piwem śmierdzi gość.
  \end{SBVerse}
  \begin{SBVerse}
    Ach, zbawco zakrzyknąłem - gdzieś się urżnął?!

    I chociaż biedak mówić już nie może,

    zachody moje były nie na próżno,

    bo wolno wybełkotał: Tam jest rożen.
  \end{SBVerse}
  \begin{SBVerse}
    I znalazłem w końcu pragnień moich przystań,

    tę oazę, gdzie wypoczywają męże.

    Gdzie kufelek piwka wzmacnia wódka czysta,

    co najlepszym w walce z kacem jest orężem.
  \end{SBVerse}
  \begin{SBVerse}
    Barman wprawną ręką kurek już odkręcił,

    szczerze wlewa mi spieniony złoty płyn.

    Tak, do piwa chyba nic mnie nie zniechęci,

    lecz rzuciłem okiem w kufel, a tu w nim...
  \end{SBVerse}

  \begin{SBChorus}
    Pierdolone muchy i komary jebane, pierdolone muchy i komary jak, ptfuj!
  \end{SBChorus}

  \begin{SBVerse}
    Komary jak \Ch{Am}{Sodoma}, mucha jak \Ch{Dm}{Gomora}!

    Pozostało mi jedyne wyjście \Ch{Am}{uciec}.

    Tak skończyło się to lato na \Ch{E7}{jeziorach},

    pozbawiły mnie owady wszelkich \Ch{Am}{uciech}.
  \end{SBVerse}

  \begin{SBChorus}
    Pierdolone muchy i komary jebane, pierdolone muchy i komary jak chuj!
  \end{SBChorus}

  \begin{SBVerse}
    Bąble dawno już otęchły, minął niesmak,

    miejski zgiełk utulił dawno w sercu żal.

    Na wycieczkach w Gerlitz wprawiam się w niemieckim,

    wołam: Ober eine flaschen beer noch mal!
  \end{SBVerse}
  \begin{SBVerse}
    Lecz, gdy wracam z tej wycieczki, dajmy na to,

    różne myśli biją w mój gorący łeb.

    Czy mi kumpel nie pokaże coś pod klapą,

    czy za klapy mnie nie złapie jakiś cep?
  \end{SBVerse}
  \begin{SBVerse}
    Gdy tak sam się po mieszkaniu w nocy tłukę,

    nie ma kogoś, kto na ramię głowę złoży,

    Zatwardziałą duszę ściśnie nieraz smutek

    i jak gdyby w serce wbił ktoś ostre noże.
  \end{SBVerse}
  \begin{SBVerse}
    Wtedy komar, każda muszka jest mi druhem,

    za ich skrzydeł brzękiem dusza mi się rwie,

    do natury, starych lip owianych puchem,

    do tych pni zwalonych, krzaków, haszczy, gdzie...
  \end{SBVerse}

  \begin{SBChorus}
    Pierdolone muchy i komary jebane, pierdolone muchy i komary jak chuj!
  \end{SBChorus}
\end{song}
