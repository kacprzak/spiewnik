\begin{song}{Jesteśmy na Wczasach}{}{}{Wojciech Młynarski}{}{}
  \begin{SBBracket}{Intro}
    \Ch{C#m}{} \Ch{F#m}{} \Ch{C#dim}{} \Ch{G#}{}
  \end{SBBracket}
  \begin{SBVerse}
    \Ch{C#m}{Za oknami noc}, w górach śniegu moc okrywa \Ch{D#7}{wszystko.}

    \Ch{F#m6}{Czort jedyny wie}, \Ch{G#}{co rzuciło mnie w to} \Ch{C#m}{uzdrowisko.}

    \Ch{G#mb5}{Na parkiecie szum}, \Ch{C#}{wczasowiczów tłum spleciony} \Ch{F#m}{gęsto.}

    \Ch{D#7}{Siedzę tutaj sam}, a przed sobą mam orkiestrę \Ch{G#}{męską.}
  \end{SBVerse}
  \begin{SBVerse}
    Typ, co szarpie bas, wie, że nadszedł czas, gdy w kimś na bańce

    Czuła struna drgnie i rozpoczną się góralskie tańce.

    Jest górala wart, taniec gdy masz fart, gdy dziewczę kwili.

    \Ch{D#7}{Z basem typ to wie}, więc szykuje się i \Ch{G#}{już po} chwili: \Ch{B7}{}
  \end{SBVerse}

  \begin{SBBracket}{Bez muzyki}
    Dla sympatycznej panny Krysi z turnusu trzeciego,

    od sympatycznego pana Waldka: pucio-pucio!
  \end{SBBracket}
  \begin{SBChorus}
    \Ch{E}{Jesteśmy na wczasach, w tych góralskich lasach,}

    W promieniach \Ch{F#m}{słonecznych} \Ch{B}{} opalamy \Ch{E6}{się.}

    Orkiestra przygrywa skocznego begina,

    To nie twoja wina, że podrywam cię...
  \end{SBChorus}
  \begin{SBVerse}
    \Ch{C#m#9}{Ta panna Krysia, panna Krysia}

    \Ch{C#m}{Królowała na turnusach nie od dzisiaj,}

    \Ch{D#7}{A każdego roku, właśnie o tej porze}

    \Ch{A#dim7}{Przyjeżdżała tu do pensjonatu} \Ch{G#}{``Orzeł''.}

    \Ch{C#m9}{Kuracjuszy rozmarzony wzrok}

    \Ch{A7}{Śledził wciąż jej każdy gest} i \Ch{G#}{krok...}
  \end{SBVerse}

  \begin{SBVerse}
    Za oknami noc, w górach śniegu moc na drzewach wisi,

    czort jedyny wie, że basista też się kocha w Krysi...

    Wie jedyny czort, co kosztuje to, by wciąż od nowa

    brać kontrabas i tłumiąc pożar krwi tak anonsować:
  \end{SBVerse}

  \begin{SBBracket}{Bez muzyki}
    Dla sympatycznej panny Krysi z turnusu trzeciego

    od sympatycznego oczywiście niewątpliwie pana Mietka: pucio-pucio!
  \end{SBBracket}
  \begin{SBChorus}
    Jesteśmy na wczasach w tych góralskich lasach,

    w promieniach słonecznych opalamy się.
  \end{SBChorus}
  \begin{SBVerse}
    A panna Krysia, panna Krysia

    z panem Mietkiem, co się tuż przed chwilą przysiadł

    przemierzała wzdłuż i wszerz parkietu przestrzeń,

    ale nigdy nie spojrzała ku orkiestrze,

    skąd basisty rozmarzony wzrok

    śledził wciąż jej każdy gest i krok.
  \end{SBVerse}
  \begin{SBChorus}
    Za oknami noc, w górach śniegu moc okrywa wszystko,

    cały turnus śpi, a wśród innych śni i nasz basista,

    że dokoła szum, na parkiecie tłum, przy czołach czoła,

    a on rzuca bas i ma w oczach blask i głośno woła:
  \end{SBChorus}

  \begin{SBBracket}{Bez muzyki}
    ``SPOKÓJ ORKIESTRA!!!

    Teraz... dla sympatycznej panny Krysi...

    z turnusu trzeciego... ode mnie...

    Panno Krysiu... kocham panią!... Wszystko...''
  \end{SBBracket}

  \begin{SBVerse}
    Ha ha ha ha ha ha ha ha!!! Co to się działo, co się działo!

    Uzdrowiska pół ze śmiechu sie skręcało

    i skręciło by do końca biednych ludzi,

    gdyby wreszcie się basista nie obudził...

    \Ch{Emaj7}{Bo miewamy często} \Ch{F#m7}{głupie} \Ch{G#m7}{sny,} \Ch{F#m7}{}

    \Ch{Emaj7}{ale potem się} \Ch{F#m7}{budzimy} \Ch{G#}{i:}
  \end{SBVerse}

  \begin{SBBracket}{Bez muzyki}
    Dla sympatycznej panny Krysi z turnusu trzeciego

    od sympatycznego, niewątpliwie, pana Waldka: pucio-pucio!
  \end{SBBracket}
\end{song}
