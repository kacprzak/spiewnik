\begin{song}{Катюша}{A-moll}{1938}{M. Isakovsky, M. Blanter}{}{}
	\begin{SBVerse}
	\Ch{Am}{Расцветали} \Ch{E7}{яблони} и груши,

	\Ch{E7}{Поплыли} \Ch{Am}{туманы} над рекой.

	\Chr{Am}{Вы}\Ch{F}{хо}\Ch{C}{дила} \Ch{Dm}{на} берег \Ch{Am}{Катюша},

	\Ch{Dm}{На} \Ch{Am}{высокий} \Ch{E7}{берег} на \Ch{Am}{крутой}.
	\end{SBVerse}
	\begin{SBVerse}
	Выходила, песню заводила

	Про степного, сизого орла,

	Про того, которого любила,

	Про того, чьи письма берегла.
	\end{SBVerse}
	\begin{SBVerse}
	Ой ты, песня, песенка девичья,

	Ты лети за ясным солнцем вслед.

	И бойцу на дальнем пограничье

	От Катюши передай привет.
	\end{SBVerse}
	\begin{SBVerse}
	Пусть он вспомнит девушку простую,

	Пусть услышит, как она поет,

	Пусть он землю бережет родную,

	А любовь Катюша сбережет.
	\end{SBVerse}
	\begin{SBVerse}
	Расцветали яблони и груши,

	Поплыли туманы над рекой.

	Выходила на берег Катюша,

	На высокий берег на крутой.
	\end{SBVerse}
\end{song}
