\documentclass[a4paper,draft]{book}
%\usepackage{fullpage}
\usepackage[chordbk]{songbook}
\usepackage{gchords}
\usepackage[utf8]{inputenc}
\usepackage{polski}
\usepackage[russian,english,polish]{babel}
% Wysokiej jakości fonty
\usepackage{lmodern}
\usepackage[T1]{fontenc}  
%\usepackage[landscape]{geometry}


%\makeTitleContents

% Styl rysowania akordów
\def\numfrets{5}
\def\xoff{2}
\def\yoff{2}
%\mediumchords
\def\chordsize{1em}

\begin{document}

%\renewcommand{\SBChordRaise}{\SBOldChordRaise}

\title{\Huge Śpiewnik}
\author{Marcin Kacprzak}
\date{\today}

\maketitle
\tableofcontents

\chapter{Po polsku}
\addcontentsline{toc}{section}{Nijak - Muchy}
\begin{song}{Muchy}{A-moll}{1984}{Nijak}{}{}
	
	\begin{SBBracket}{Intro}
	Oj, \Chr{Am}{było}, było \Chr{E7}{lato}, oj, było, było \Ch{Am}{lato}...

	Oj, było, było lato, oj, było, było lato...
	\end{SBBracket}

	\begin{SBVerse}
	Z mych wakacji to wspomnienia będą jednych,

	moich uwag zbiór świadczący o tym, że

	świat owadów może być nad wyraz wredny

	i w dodatku prześladuje ciągle mnie.
	\end{SBVerse}
	\begin{SBVerse}
	Będąc szczupłym jeszcze latem raz poznałem

	piękne dziewczę wykształcone tu i tam.

	Że bez mamy wypoczywa tak mniemałem,

	więc zdobycia jej uknułem chytry plan.
	\end{SBVerse}
	\begin{SBVerse}
	Przedstawiłem się nazajutrz jej na plaży,

	mówiąc: Proszę, tu na kocu miejsce mam.

	I choć słońce dwa pośladki moje smaży,

	w oczy patrzę jej, Don Chuana minę mam.
	\end{SBVerse}
	\begin{SBVerse}
	Kremem prosi, bym jej plecy wysmarował,

	więc przylgnąłem jak pijawka do jej ciała.

	Nagle mówi: Wie pan, żar mnie zmitygował,

	w chłodnym gaju bym pospacerować chciała.
	\end{SBVerse}
	\begin{SBVerse}
	Ach, bogowie! Czegóż więcej pragnąć mogłem,

	brzuch wciągnąłem, wyprężyłem nagi tors.

	I w najbardziej gęste chaszcze ją powiodłem,

	ręką obejmując piękny, jędrny gors.
	\end{SBVerse}
	\begin{SBVerse}
	Spacer zmęczył dziewczę pośród pni zwalonych,

	lekko wsparła głowę na ramieniu mym,

	Ja wargami szukam ust jej rozchylonych,

	już znalazłem, już chcę wpić się, gdy wtem...
	\end{SBVerse}

	\begin{SBChorus}
	Pierdolone muchy i komary jebane, pierdolone muchy i komary jak, ptfuj!
	\end{SBChorus}

	\begin{SBVerse}
	Co się stało pyta - boi się pan muszek?

	Nie odrzekłem mam nerwowy tik.

	Znów dziewiczej nieśmiałości lody kruszę,

	już puls szybciej bije jej, wtem nagle ...~bzyk.
	\end{SBVerse}

	\begin{SBChorus}
	Pierdolone muchy i komary jebane, pierdolone muchy i komary jak chuj!
	\end{SBChorus}

	\begin{SBVerse}
	Ciało w bąblach jest na plaży niezbyt piękne,

	zniechęcony całkiem do miłosnych uciech,

	Wdziałem koszulinę lekką i spodenki,

	zapragnąłem spić zimnego piwka kufel.
	\end{SBVerse}
	\begin{SBVerse}
	A, lecz wiadomo, lato, żar, skąd piwo zimne?

	Nie Miami Beach to nawet i nie Soczi.

	I choć Heineken mi się śnił, czy jakieś inne,

	to nawet w słupskim piwie chciałbym usta zmoczyć.
	\end{SBVerse}
	\begin{SBVerse}
	Poszukiwania prowadziłem przez dzień cały,

	już zrezygnować chciałem, miałem tego dość,

	Wtem widzę: facet się zatacza znietrzeźwiały,

	podchodzę, wącham piwem śmierdzi gość.
	\end{SBVerse}
	\begin{SBVerse}
	Ach, zbawco zakrzyknąłem - gdzieś się urżnął?!

	I chociaż biedak mówić już nie może,

	zachody moje były nie na próżno,

	bo wolno wybełkotał: Tam jest rożen.
	\end{SBVerse}
	\begin{SBVerse}
	I znalazłem w końcu pragnień moich przystań,

	tę oazę, gdzie wypoczywają męże.

	Gdzie kufelek piwka wzmacnia wódka czysta,

	co najlepszym w walce z kacem jest orężem.
	\end{SBVerse}
	\begin{SBVerse}
	Barman wprawną ręką kurek już odkręcił,

	szczerze wlewa mi spieniony złoty płyn.

	Tak, do piwa chyba nic mnie nie zniechęci,

	lecz rzuciłem okiem w kufel, a tu w nim...
	\end{SBVerse}

	\begin{SBChorus}
	Pierdolone muchy i komary jebane, pierdolone muchy i komary jak, ptfuj!
	\end{SBChorus}

	\begin{SBVerse}
	Komary jak \Ch{Am}{Sodoma}, mucha jak \Ch{Dm}{Gomora}!

	Pozostało mi jedyne wyjście \Ch{Am}{uciec}.

	Tak skończyło się to lato na \Ch{E7}{jeziorach},

	pozbawiły mnie owady wszelkich \Ch{Am}{uciech}.
	\end{SBVerse}

	\begin{SBChorus}
	Pierdolone muchy i komary jebane, pierdolone muchy i komary jak chuj!
	\end{SBChorus}

	\begin{SBVerse}
	Bąble dawno już otęchły, minął niesmak,

	miejski zgiełk utulił dawno w sercu żal.

	Na wycieczkach w Gerlitz wprawiam się w niemieckim,

	wołam: Ober eine flaschen beer noch mal!
	\end{SBVerse}
	\begin{SBVerse}
	Lecz, gdy wracam z tej wycieczki, dajmy na to,

	różne myśli biją w mój gorący łeb.

	Czy mi kumpel nie pokaże coś pod klapą,

	czy za klapy mnie nie złapie jakiś cep?
	\end{SBVerse}
	\begin{SBVerse}
	Gdy tak sam się po mieszkaniu w nocy tłukę,

	nie ma kogoś, kto na ramię głowę złoży,

	Zatwardziałą duszę ściśnie nieraz smutek

	i jak gdyby w serce wbił ktoś ostre noże.
	\end{SBVerse}
	\begin{SBVerse}
	Wtedy komar, każda muszka jest mi druhem,

	za ich skrzydeł brzękiem dusza mi się rwie,

	do natury, starych lip owianych puchem,

	do tych pni zwalonych, krzaków, haszczy, gdzie...
	\end{SBVerse}

	\begin{SBChorus}
	Pierdolone muchy i komary jebane, pierdolone muchy i komary jak chuj!
	\end{SBChorus}
\end{song}

\begin{song}{Keja}{A-moll}{1971}{Jerzy Porębski}{}{}
	\begin{SBVerse}
	\Ch{Am}{Gdyby} ktoś przyszedł i powiedział - \Ch{G}{stary} czy masz \Ch{Am}{czas}?

	\Ch{C}{Potrzebuje} do załogi \Ch{G}{jakąś} nowa \Ch{C}{twarz}

	\Ch{C7}{Amazonka}, Wielka \Ch{F}{rafa}, oceany \Ch{Dm}{trzy}

	\Ch{Am7}{Rejs} na całość: rok, \Ch{E}{dwa} lata - \Ch{Am}{odpowiedziałbym}:
	\end{SBVerse}

	\begin{SBChorus}
	\Ch{Am}{Gdzie} ta keja, a \Ch{E7}{przy} niej ten \Ch{Am}{jacht}

	\Ch{C}{Gdzie} ta \Ch{G}{koja} wymarzona w \Ch{C}{snach} \Ch{C7}{}

	\Ch{Gm}{Gdzie} te wszystkie \Ch{A7}{sznurki} \Chr{Dm}{od} \Ch{A7}{tych} \Ch{Dm}{szmat}

	\Ch{Am}{Gdzie} ta brama \Ch{E7}{na szeroki} \Ch{Am}{świat}.
	\end{SBChorus}

	\begin{SBChorus}
	Gdzie ta keja, a przy niej ten jacht

	Gdzie ta koja wymarzona w snach

	W każdej chwili płynę w taki rejs

	Tylko gdzie to jest, gdzie to jest?

	\end{SBChorus}

	\begin{SBVerse}
	Gdzieś na dnie wielkiej szafy leży ostry nóż

	Stare dżinsy wystrzępione impregnuje kurz

	W kompasie igła zardzewiała, lecz kierunek znam

	Biorę wór na plecy i przed siebie gnam.
	\end{SBVerse}

	\begin{SBChorus}
	Gdzie ta keja, a przy niej ten jacht...
	\end{SBChorus}

	\begin{SBVerse}
	Przeszły lata zapyziałe, rzęsą zarósł staw

	A na przystani czółno stało - kolorowy paw

	Zaokrągliły się marzenia, wyjałowiał step

	Lecz dalej marzy o załodze ten samotny łeb.
	\end{SBVerse}

	\begin{SBChorus}
	Gdzie ta keja, a przy niej ten jacht...
	\end{SBChorus}
\end{song}

\addcontentsline{toc}{section}{Big CyC}
\addcontentsline{toc}{subsection}{Każdy Facet To Świnia}
\begin{song}{Każdy Facet To Świnia}{G}{2002}{Big Cyc}{}{}
	\begin{SBVerse}
	\Ch{G}{Jak} zwykle znów nie robisz nic

	\Ch{Em}{Gazetę} czytasz cały dzień

	\Ch{C}{Łaskawie} czasem obiad zjesz

	\Ch{Am}{Po domu} snujesz się \Ch{D}{jak cień}

	\Ch{G}{Ty z} kolegami wolisz pić

	\Ch{Em}{Niż z} moją mamą ciasto piec

	\Ch{C}{I zamiast} dzieckiem zająć się

	\Ch{Am}{Musiałeś} znowu \Chr{D}{wyjść} na mecz

	\Ch{C}{To nie} jest miłość, \Ch{Hm}{lecz} ja kocham Cię

	\Ch{C}{Nie jestem} świnią, \Ch{Am}{choć} ty tego \Ch{D}{chcesz}.
	\end{SBVerse}
	\begin{SBChorus}
	\Ch{G}{Facet} to świnia

	\Ch{Em}{Mówisz}, że ty o tym wiesz

	\Ch{C}{Choć} ja się staram jak mogę

	\Ch{Am}{Przez} całe życie słyszę ten \Ch{D}{tekst} / x 2
	\end{SBChorus}
	\begin{SBVerse}
	Ty w telewizor gapisz się

	A do kościoła chodzisz sam

	I nigdy nie przytulisz mnie

	W łazience znowu cieknie kran

	Gdy w nocy czujesz się jak lew

	To obręcz ściska moją skroń

	No kiedy wreszcie puścisz mnie

	Migrena to najlepsza broń

	To nie jest miłość, lecz ja kocham Cię

	Nie jestem świnią, choć ty tego chcesz.
	\end{SBVerse}
	\begin{SBChorus}
	Facet to świnia. . . itd. / x 2
	\end{SBChorus}
	\begin{SBVerse}
	O samochodach mówisz wciąż

	Do dziewczyn ślinisz się jak pies

	Ty życie zmarnowałeś mi

	Od kogo jest ten SMS?

	I chociaż oszukujesz mnie

	Ja lubię twój szelmowski śmiech

	Bez ciebie nudny byłby świat

	Bo facet to jest dobra rzecz

	To nie jest miłość, lecz ja kocham Cię

	Nie jestem świnią, choć ty tego chcesz.
	\end{SBVerse}
	\begin{SBChorus}
	Facet to świnia. . . itd. / x 3
	\end{SBChorus}
\end{song}

\addcontentsline{toc}{subsection}{Dramat fryzjerski}
\begin{song}{Dramat fryzjerski}{}{}{Big Cyc}{}{}
	\begin{SBVerse}
	\Ch{H}{Nikt} nie miał włosów takich jak ty

	\Ch{G#m}{Na punkcie} tych kłaków zgłupiałem

	\Ch{C#m}{Były} pachnące jak świeże bzy

	\Ch{F#}Dostałem świra, zemdlałem
	\end{SBVerse}

	\begin{SBVerse}
	Ty ocuciłaś mnie swoim uściskiem

	Myślałem, że już nie żyję

	Złapałem twe kłaki poczułem, że śliskie

	"Zocha przede mną coś kryjesz".
	\end{SBVerse}

	\begin{SBChorus}
	\Ch{E}{Czy} ty \Ch{Em}{wiesz}, że mam \Ch{H}{łupież}, \Ch{G#m}{łupież}

	\Ch{E}{Czy} ty \Ch{Em}{wiesz}, że mam \Ch{H}{łupież}
	\end{SBChorus}

	\begin{SBVerse}
	Oblałem twe pukle denaturatem

	Potem moczyłem je w zupie

	Skrecałem twój warkocz jak mokrą szmatę

	Żeby wytępić łupież
	\end{SBVerse}

	\begin{SBVerse}
	Wzięłem w swe dłonie pompkę gumową

	Rąbnąłem cię miedzy oczy

	Wyglądasz teraz jak Shinead O'Connor

	Twój wygląd jest nadal uroczy
	\end{SBVerse}

	\begin{SBChorus}
	Ref.
	\end{SBChorus}

	\begin{SBVerse}
	Ostatnia moja deska ratunku

	To list napisna do "Jestem"

	Tutaj Krzyś Skiba pełen szacunku

	Poradźcie, bo wpadnę w depresję
	\end{SBVerse}
	\begin{SBVerse}
	Jest taka sprawa, ona ma łupież

	A ja jestem wnerwiony

	Przyszła odpowiedź: "To proste chłopcze,

	Zrób z tego serek topiony.
	\end{SBVerse}

	\begin{SBChorus}
	Ref.
	\end{SBChorus}
\end{song}


\addcontentsline{toc}{section}{Alosza Awdiejew}
\begin{song}{Kołysanka Pijacka}{A-moll}{1994}{Alosza Awdiejew, Włodzimerz Wysocki}{}{}
	\begin{SBVerse}
	\Ch{E}{Jak} po naszemu to wypiliśmy \Ch{Am}{niewiele}

	\Ch{G}To nie z kieliszka, no powiedz \Ch{C}{Miszka} \Ch{A7}{}

	\Ch{Dm}{I gdyby} wódki nie pędzono jakich \Ch{Am}{śmieci}

	\Ch{E}{To} nie zaszkodziłby nam ten literek \Chr{Am}{trzeci}
	\end{SBVerse}
	\begin{SBVerse}
	Ten pierwszy pilismy w kąciku koło lady

	To był początek jakby bez wiary

	A trzecią w parku za zakrętem koło urny

	Gdzie nie pamięta, film się urwał durny
	\end{SBVerse}
	\begin{SBVerse}
	Z butelki piłem i w dodatku nic nie jadłem

	I trudno było jak cholera lecz nie padłem

	A kiedy władza wreszcie nas zdybała

	To w każdym z nas po litrze już pływało
	\end{SBVerse}
	\begin{SBVerse}
	Popatrz, popatrz Miszka, nas szacunkiem otaczają

	Popatrz podwożą, popatrz wsadzają

	A rano też, jak szanowanych ludzi

	Nie jakiś tam kogut, a sierżant obudzi
	\end{SBVerse}
	\begin{SBVerse}
	Prześpijmy się spokojnie to nie praca

	Mam trochę forsy, złagodzimy kaca
	\end{SBVerse}
	\begin{SBBracket}{2x}
	\Ch{Dm}{Bo} w życiu mało jest dobrego wiele \Chr{Am}{złego}

	\Ch{E}{A co} tam gadać, no śpij \Ch{Am}{kolego} \Ch{A}{}
	\end{SBBracket}
\end{song}

\begin{song}{Murka}{A-moll}{}{Alosza Awdiejew}{}{}
	\begin{SBVerse}
	\Ch{Am}{Kto} nie zna w Odessie \Ch{E7}{bandy} słynnej  \Ch{Am}{Murki}

	Jest w niej rzezi\Ch{H7}{mieszków} cała \Ch{E7}{moc}

	\Ch{Am}{Kradną} zabijają \Ch{A7}{towar} \Ch{Dm}{przemycają}

	\Ch{Am}{Śledzą} ich glinia\Ch{E7}{rze} dzień i \Ch{Am}{noc}
	\end{SBVerse}
 
	\begin{SBVerse}
	Często na melinie Murka przymawiała

	I za nią bandyci szli jak w dym

	Piękna ona była bandzie przewodziła

	Wśród złodziei zawsze wiodła prym
	\end{SBVerse}

	\begin{SBVerse}
	Raz do skoku szliśmy postanowiliśmy 

	Na jednego do lokalu wpaść

	A tam siedzi Murka z frajerem tajniakiem

	A z kieszeni kurtki sterczy gnat
	\end{SBVerse}
	 
	\begin{SBVerse}
	Po co Ci to Murka źle ci było z nami

	A czy Ci to czego było brak

	Co Cię podkusiło zwąchać się z glinami    

	Czemuś Murka nas zdradziła tak
	\end{SBVerse}
	 
	\begin{SBVerse}
	Dawniej toś Ty miała pantofelki złote

	I karakułowe szuby dwie

	Dzisiaj brudne szmaty same dziury łaty

	Bo nie jesteś już w ferajnie nie
	\end{SBVerse}
	 
	\begin{SBVerse}
	Żegnaj moja Murka żegnaj mi kochana

	Żegnaj Murka nam na wieki już
	\end{SBVerse}
	\begin{SBBracket}{2x}
	Sprowadziłaś gliny do naszej meliny

	To powąchaj teraz fiński nóż
	\end{SBBracket}

\end{song}

\addcontentsline{toc}{section}{T-Raperzy znad Wisły}
\begin{song}{Spotkajmy się}{C}{}{T-Raperzy znad Wisły}{}{}
	\begin{SBChorus}
	Spotkajmy się je, je, je x4
	\end{SBChorus}
	\begin{SBVerse}
	Spotkajmy \Ch{Dm}{się,} spotkajmy \Ch{G}{się}

	Na jedną \Ch{C}{chwilę}, może \Ch{Am}{dwię}

	A kiedy już spotkamy się

	Przywitam ciebie, a ty mnię
	\end{SBVerse}
	\begin{SBChorus}
	Spotkajmy się je, je, je x4
	\end{SBChorus}
	\begin{SBVerse}
	Spotkajmy się, spotkajmy się

	Na rogu albo byle gdzię

	A kiedy już spotkamy się

	Przyrzekam, że nie będzie źlę
	\end{SBVerse}
	\begin{SBChorus}
	Spotkajmy się je, je, je x4
	\end{SBChorus}
	\begin{SBVerse}
	Spotkajmy się, spotkajmy się

	Bo razem raźniej, każdy wię

	I duża radość będzie w nas

	Kiedy spędzimy razem czas
	\end{SBVerse}
	\begin{SBChorus}
	Spotkajmy się je, je, je x4
	\end{SBChorus}
	\begin{SBVerse}
	Spotkajmy się, spotkajmy się

	Choćby na wczasach w FWP

	A gdy spotkania przyjdzie kres

	Urońmy choć po kilka łez
	\end{SBVerse}
	\begin{SBVerse}
	Mam randkę dziś punkt szósta

	Dlatego już idę odświeżyć usta

	Pięć nabyć róż
	\end{SBVerse}
	\begin{SBChorus}
	Spotkajmy się je, je, je x4
	\end{SBChorus}
	\begin{SBVerse}
	Spotkajmy się, spotkajmy się

	I proszę, tylko nie mów nię

	Bo może już następny raz

	Spotkanie to połączy nas

	Spotkajmy się je, je, je...
	\end{SBVerse}
\end{song}

\begin{song}{Kraina łagodności}{C}{}{T-Raperzy znad Wisły}{}{}
	\begin{SBChorus}
	\Ch{C}{Dabudibuda} o \Ch{Dm}{naturze} ludzkiej \Ch{G}pieśń jest \Ch{Am}{ta}

	Dabudibuda o naturze ludzkiej pieśń jest ta
	\end{SBChorus}
	\begin{SBVerse}
	\Ch{Am}{Zwiedzawszy} areały pustych dusz przestrzeni

	\Ch{Dm}{Rzeźbimy} w glinie uczuć ponadczasowy dzban

	\Ch{G}{I myśli} naszych prądy jak rogi dwóch jeleni

	\Ch{Am}{ku sobie} rwą na przeciw \Ch{E}{nim rui} poczną tan
	\end{SBVerse}
	\begin{SBChorus}
	Dabudibuda o naturze ludzkiej pieśń wciąż trwa

	Dabudibuda o naturze ludzkiej pieśń wciąż trwa
	\end{SBChorus}
	\begin{SBVerse}
	Dyspozytorze chuci, kosmiczny szarlatanie

	Neurony nam splątawszy, podniety dałeś kwiat

	I w lubieżności wannie totalne urządź pranie

	Puste przebiegi wstydu wpisz w harmonogram strat
	\end{SBVerse}
	\begin{SBChorus}
	Dabudibuda o naturze ludzkiej pieśń wciąż trwa

	Dabudibuda o naturze ludzkiej pieśń wciąż trwa
	\end{SBChorus}
	\begin{SBVerse}
	Zwolniwszy metabolizm tęsknoty w naszych łonach

	Siejemy w ciała jamach poplony starych snów

	Odwieczne dylematy co psyche, a co soma

	Rozszarpią nam świadomość jak para wściekłych psów
	\end{SBVerse}
	\begin{SBChorus}
	Dabudibuda o naturze ludzkiej pieśń wciąż trwa

	Dabudibuda o naturze ludzkiej pieśń wciąż trwa
	\end{SBChorus}
	\begin{SBVerse}
	Niech buczy transformator sieci ciał naszych splotu

	Jak puchacz w noc posępną, co myszy dojrzał cień

	Ta mysz źródłem podniety jest dla miliarda kotów

	Więc brudnych myśli ojców bez żalu wyrżnij w pień
	\end{SBVerse}
	\begin{SBChorus}
	Dabudibuda ona wciąż kłopoty z głową ma

	Dabudibuda ona wciąż kłopoty z głową ma
	\end{SBChorus}
\end{song}

\addcontentsline{toc}{section}{Mumio}
\begin{song}{Kubeczek}{A-moll}{}{Mumio}{}{}
	\begin{SBVerse}
	\Ch{E}{Skończyła} się już noc

	\Ch{Am}{żaluzje} się rozjechały

	na głowę naciągam koc

	jest dla mnie za mały

	wystają mi nogi

	jest mi zimno w nie

	nie kocham ciebie

	ty nie kochasz mnie

	a przecież było nam

	tak jak nie wiem gdzie /impro.../
	\end{SBVerse}
	\begin{SBChorus}
	\Ch{E}{Jak to}, jak to się \Ch{Am}{stało},

	\Ch{E}{że mi} wypiłaś

	\Ch{Am}{moje} kakao

	\Ch{Dm}{przecież} to mój \Ch{Am}{kubeczek}

	\Ch{E}{to mój} kubeczek

	\Ch{Am}{z wiewiórką} jest
	\end{SBChorus}
\end{song}

\begin{song}{Jak dobrze być Barankiem}{D}{}{}{}{}
	\begin{SBVerse}
	\Ch{D}Jak dobrze być barankiem

	i wstawać sobie \Ch{A}{rankiem}

	wybiegać na polanke

	i spiewać sobie \Ch{D}{tak}
	\end{SBVerse}
	\begin{SBChorus}
	\Ch{D}bee, bee, bee, kopytka niosą \Ch{A}{mnie}

	\Ch{A}bee, bee, bee, kopytka niosą \Ch{D}{mnie}
	\end{SBChorus}
	\begin{SBVerse}
	How good to be baranek

	and wake up sobie ranek

	and running to polanek

	and singing just like that
	\end{SBVerse}
	\begin{SBChorus}
	be, be, be kopytka taking me

	be, be, be kopytka taking me
	\end{SBChorus}
	\begin{SBVerse}
	So gut zu sein baranek

	aufsteigen sobie ranek

	und gehen nach polanek

	und singen so wie als
	\end{SBVerse}
	\begin{SBChorus}
	bich, bich, bich, kopytka tragen mich

	bich, bich, bich, kopytka tragen mich
	\end{SBChorus}
	\begin{SBVerse}
	C’est bien d’etre un baranek

	et se berer sobie ranek

	et courir sur polanek

	et chanter comme sa
	\end{SBVerse}
	\begin{SBChorus}
	boit, boit, boit, kopytka portent moi

	boit, boit, boit, kopytka portent moi
	\end{SBChorus}
	\begin{SBVerse}
	Bene essere baranek

	alzarsi sobie ranek

	corere na polanek

	e cantare me si
	\end{SBVerse}
	\begin{SBChorus}
	be, be, be, kopytka portano me

	be, be, be, kopytka portano me
	\end{SBChorus}
\end{song}



\addcontentsline{toc}{section}{Breakout}
\begin{song}{Kiedy byłem małym chłopcem}{A-moll}{1971}{Breakout}{}{}
	\begin{SBVerse}
	Kiedy byłem,

	Kiedy byłem małym chłopcem, hej,

	Wziął mnie ojciec,

	Wziął mnie ojciec i tak do mnie rzekł:

	Najważniejsze co się czuje,

	Słuchaj zawsze głosu serca, hej.
	\end{SBVerse}
	\begin{SBVerse}
	Kiedy byłem,

	Kiedy byłem dużym chłopcem, hej,

	Wziął mnie ojciec,

	Wziął mnie ojciec i tak do mnie rzekł:

	Głosem serca się nie kieruj,

	Tylko forsa ważna w życiu jest.
	\end{SBVerse}
	\begin{SBVerse}
	Wicher wieje,

	Wicher słabe drzewa łamie, hej,

	Wicher wieje,

	Wicher silne drzewa głaszcze, hej.

	Najważniejsze to być silnym,

	Wicher silne drzewa głaszcze, hej.
	\end{SBVerse}
\end{song}

\addcontentsline{toc}{section}{Soyka}
\begin{song}{Czas nas uczy pogody}{C-moll}{1984}{Jacek Cygan, Krzesimir Dębski}{}{}

	\begin{SBBracket}{2x}
	\Ch{Cm}{} \Ch{G}{}
	\end{SBBracket}
 
	\begin{SBVerse}
	\Ch{Cm}{Widziałem} wiatr o \Ch{Gm}{siwych} włosach, \Ch{G#}{roznosił} spokój wśród \Ch{Eb}{pól},
	 
	\Ch{Fm}{W ciepłe} babie \Ch{Cm}{lato} kości \Ch{B}{grzał.}

	\Ch{Cm}{A innym} razem \Ch{Gm}{lasy} kosił, \Ch{G#}{spadał} ostrzem z \Ch{Eb}{gór},

	\Ch{Fm}{Młody} był, \Ch{Gm}{bogiem} był i \Ch{G#}{gnał} \Ch{B}{} wolny \Ch{Eb}{tak.}
	\end{SBVerse}

	\begin{SBChorus}
	\Ch{B}{Wiele} dni, \Ch{G#7+}{wiele} lat \Ch{Fm7}czas nas uczy \Ch{Gm7}{pogody},                      

    	\Ch{G#}Zaplącze drogi, \Ch{Fm7}{pomyli} prawdy, nim \Ch{G#}{zboże} \Ch{B}{oddzieli} od \Ch{Eb}{trawy}.

    	\Ch{B}{Bronisz} się, \Ch{G#7+}{siejesz} wiatr, \Ch{Fm7}{myślisz}: jestem tak \Ch{Gm7}{młody},
 
    	\Ch{G#}{Czas} nas uczy \Ch{Eb}{pogody}, tak od \Ch{F}{lat},  \Ch{G7}{tak} od \Ch{C}{lat}. \Ch{C#}{} \Ch{C}{}
	\end{SBChorus} 

	\begin{SBChorus}
    	\Ch{C9}{Ilu} ludzi \Ch{Dm7}{czas} \Ch{G0}{wyleczył} z \Ch{C7}{ran},

    	\Ch{G9}{Zamienił} w \Ch{G4}{spokój} \Ch{Eb}{bu}\Ch{F}{rzę} \Ch{G}{krwi}. \Ch{G#}{} \Ch{G}{}
 
    	\Ch{C9}{Pewnie} kiedyś \Ch{Dm7}{nam}, pod \Ch{G0}{jesień} \Ch{C7}{tak},

    	Też \Ch{G9}{czoło} \Ch{G4}{wypogodzi} \Ch{Eb}{i}  \Ch{F}{wygładzi} \Ch{G}{brwi}. \Ch{G#}{} \Ch{G}{}
	\end{SBChorus} 

	\begin{SBVerse}
	\Ch{Cm}{Widziałem} dni w \Ch{Gm}{muzeach} sennych o \Ch{G#}{wnętrzach} zimnych jak \Ch{Eb}{mrok},

	\Ch{Fm}{Starsi} ludzie w \Ch{Cm}{rogach} wielkich \Ch{B}{sal}.

	\Ch{Cm}{Księgi} pięknych \Ch{Gm}{myśli} pełne \Ch{G#}{pokrył} gruby \Ch{Eb}{kurz},

	\Ch{Fm}{Herbaty} smak, \Ch{Cm}{kapci} miękkich \Ch{G#}{szum}, \Ch{B}{} spokój \Ch{Eb}{serc.}
	\end{SBVerse}

	\begin{SBChorus}
	Wiele dni, wiele lat… tak od lat 
	\end{SBChorus}

\end{song}

\chapter{Po angielsku}
%\begin{otherlanguage}{english}
\addcontentsline{toc}{section}{Bob Seger}
\begin{song}{Turn the Page}{E-moll}{1973}{Bob Seger}{}{}

	\begin{SBVerse}
	\Ch{Em}{On a} long and lonely highway east of omaha

	you can \Ch{D}{listen} to the engine, moanin out as one long song

	you can \Ch{A}{think} about the woman, or the girl you knew the night \Ch{Em}{before}
	\end{SBVerse}
	\begin{SBVerse}
	and your thoughts will soo be wandering the way they always do

	when your riding sixteen hours and there's nothing much to do

	you don't feel much like travelin', you just wish the trip was through
	\end{SBVerse}
	\begin{SBChorus}
	but here I \Ch{D}{am}, on the road \Ch{Em}{again}

	here I \Ch{D}{am}, up on the \Ch{Em}{stage}

	here I \Ch{D}{go}, playing the \Ch{A}{star} again

	there I \Ch{C}{go}, \Ch{D}{turn} the \Ch{Em}{page}
	\end{SBChorus}
	\begin{SBVerse}
	you walk into a restaraunt, strung out from the road

	and you feel the eyes opon you, as your shaking off the cold

	you pretend it doesn't bother you, but you just want to explode
	\end{SBVerse}
	\begin{SBVerse}
	sometimes you hear 'em talkin', other times you can't

	all the same 'ole cliche's is that a woman or a man

	and you always seem outnumbered, you dare not make a stand
	\end{SBVerse}
	\begin{SBChorus}
	CHORUS
	\end{SBChorus}
	\begin{SBVerse}
	out there in the spotlight, your a million miles away

	every ounce of energy, you try to give away

	and the sweat pours from your body, like the music that you play
	\end{SBVerse}
	\begin{SBVerse}
	later on that evening, as you lie awake in bed

	echos of the amplifiers, ringin in your head

	and you smoke the days last cigarette, remembering what you said
	\end{SBVerse}
	\begin{SBChorus}
	CHORUS 2x
	\end{SBChorus}
\end{song}
%\begin{otherlanguage}

\addcontentsline{toc}{section}{Ed Bruce}
\begin{song}{Mammas Don't Let Your Babies Grow Up to Be Cowboys}{C}{1975}{Ed and Patsy Bruce}{}{}
	\begin{SBChorus}

	\Ch{C}{Mammas} don't let your babies grow up to be \Ch{F}{cowboys}

	Don't \Ch{G7}{let} them pick guitars and drive in old trucks

	Make 'em be doctors and lawyers and \Ch{C}{such}

	Mammas don't let your babies grow up to be \Ch{F}{cowboys}

	They'll \Ch{G7}{never} stay home and they're always alone

	Even with someone they \Ch{C}{love}
	\end{SBChorus}

	\begin{SBVerse}
	Cowboys ain't easy to love and they're harder to \Ch{F}{hold}

	And \Ch{G7}{they'd} rather give you a song than diamonds or \Ch{C}{gold}

	Lone Star belt buckles and old faded Levis

	And each \Ch{F}{night} begins a new day

	And if \Ch{G7}{you} don't understand him and he don't die young

	He'll probably just ride \Ch{C}{away}
	\end{SBVerse}
	\begin{SBChorus}
	CHORUS
	\end{SBChorus}

	\begin{SBVerse}

	A cowboy loves smokey old pool rooms and clear mountain \Ch{F}{mornings}

	\Ch{G7}{Little} warm puppies and children and girls of the \Ch{C}{night}

	Them that don't know him won't like him

	And them that do \Ch{F}{sometimes} won't know how to take him

	He's not \Ch{G7}{wrong} he's just different and his pride won't let him

	Do things to make you think he's \Ch{C}{right}
	\end{SBVerse}
	\begin{SBChorus}
	CHORUS
	\end{SBChorus}
\end{song}

\begin{otherlanguage}{english}
\addcontentsline{toc}{section}{Eric Clapton}
\begin{song}{Tears In Heaven}{A}{1992}{Eric Clapton}{}{}

	\Ch{A}{} \Ch{E}{} \Ch{F#m}{} \Ch{D}{} \Ch{E}{} \Ch{A}{}
	\begin{SBVerse}
	\Ch{A}{Would} you \Ch{E}{know} my \Ch{F#m}{name}

	\Ch{D}{if} I \Ch{A}{saw} you in \Ch{E}{heaven}?

	Would it be the same

	if I saw you in heaven?

	\Ch{F#m}{I must} be \Ch{C#}{strong}  \Ch{Em}{and} carry \Ch{F#}{on},

	Cause  I  \Ch{Hm}{know} I don't \Ch{E}{belong} here in heaven.
	\end{SBVerse}
	\Ch{A}{} \Ch{E}{} \Ch{F#m}{} \Ch{D}{} \Ch{E}{} \Ch{A}{}
	\begin{SBVerse}
	Would you hold my hand

	if I saw you in heaven?

	Would you help me stand

	if I saw you in heaven?

	I'll find my way, through night and day

	Cause  I  know I just can stay here in heaven.
	\end{SBVerse}
	\Ch{A}{} \Ch{E}{} \Ch{F#m}{} \Ch{D}{} \Ch{E}{} \Ch{A}{}


	\Ch{C}{Time} can \Ch{G}{bring} you \Ch{Am}{down}, time can \Ch{D}{bend} your \Ch{G}{knees}.\Ch{D}{} \Ch{Em}{}

	\Ch{C}{Time} can \Ch{G}{break} your \Ch{Am}{heart}, have you \Ch{D}{begging} \Ch{G}{please}, \Ch{D}{begging} \Ch{E}{please}.

	\begin{SBOpGroup}
	SOLO: 2x
	\end{SBOpGroup}
	\begin{SBVerse}
	\Ch{F#m}{Beyond} the \Ch{C#}{door}, \Ch{Em}{there's} peace I'm \Ch{F#}{sure}

	And  I   \Ch{Hm}{know} there'll be no \Ch{E}{more}   tears in heaven.
	\end{SBVerse}

	\Ch{A}{} \Ch{E}{} \Ch{F#m}{} \Ch{D}{} \Ch{E}{} \Ch{A}{}
	\begin{SBOpGroup}
	Repeat first verse
	\end{SBOpGroup}
\end{song}
\end{otherlanguage}

\begin{song}{Horse with no name}{E-moll}{1973}{America}{}{}
	\begin{SBVerse}
	\Ch{Em}{On} the first part of the \Ch{D6/9}{journey}

	I was lookin at all the life

	There were plants and birds and rocks and things

	There were sand and hills and rings
	\end{SBVerse}

	\begin{SBVerse}
	The first thing I met was a fly with a buzz

	and the sky with no clouds

	the heat was hot and the ground was dry

	but the air was full of sound
	\end{SBVerse}

	\begin{SBChorus}
	\Ch{Em9}{I've} been through the desert on a \Ch{Dm9}{horse} with no name

	it felt good to be out of the rain

	in the desert you can remember your name

	'cause there ain't no one for to give you no pain

	la la la  la lalala   la la  la  la la
	\end{SBChorus}

	\begin{SBVerse}
	After two days in the desert sun

	my skin began to turn red

	After three days in the desert fun

	I was looking at a river bed

	And the story it told of a river that flowed

	made me sad to think it was dead
	\end{SBVerse}

	\begin{SBChorus}
	CHORUS
	\end{SBChorus}

	\begin{SBVerse}
	After nine days I let the horse run free

	'cause the desert had turned to sea

	there were plants and birds and rocks and things

	there were sand and hills and rings

	The ocean is a desert with it's life underground

	and the perfect disguise above

	Under the cities lies a heart made of ground

	but the humans will give no love
	\end{SBVerse}

	\chords{
		\chord{t}{p2,n,n,p2,n,n}{D6/9}
		\chord{t}{n,n,n,p2,p2,n}{Dm9}
	}

\end{song}

\addcontentsline{toc}{section}{Stereophonics}
\begin{song}{Long Way Round}{C}{2004}{Stereophonics}{}{}

	\begin{SBVerse}
	\Ch{C}{Remember} me my \Ch{G}{love}

	\Ch{Am}{I'm} the one ya \Ch{F}{dreaming of}

	I'm going for a ride

	I'll keep you warm inside
	\end{SBVerse}

	\begin{SBVerse}
	Gonna roll up the sidewalk

	Gonna tear up the ground

	Coming round to meet you

	The long way round
	\end{SBVerse}

	\begin{SBVerse}
	Sooner or later

	I'll get me off this track

	Gotta do what it is that I do

	Then I'm coming back
	\end{SBVerse}

	\begin{SBVerse}
	Got the sun in my face

	Sleeping rough off the road

	I'll tell you all about it

	When I get home
	\end{SBVerse}

	\begin{SBVerse}
	Gonna roll up the sidewalk

	Gonna tear up the ground

	Coming round to meet you

	The long way round
	\end{SBVerse}
\end{song}

\addcontentsline{toc}{section}{Bob Marley}
\begin{song}{Redemption Song}{G}{1980}{Bob Marley}{}{}

	\begin{SBVerse}
	Old \Ch{G}{pirates}, yes, they rob \Ch{Em7}{I}

	\Ch{C}{sold} I to the \Ch{C/B}{merchant} \Ch{Am}{ships}

	\Ch{G}{Minutes} after they took \Ch{Em}{I}

	\Ch{C}{from} the \Ch{C/B}{bottomless} \Ch{Am}{pit.}

	But my \Ch{G}{hand} was \Ch{Em7}{made} \Ch{C}{strong}

	by the \Ch{C}{hand} of \Chr{C/B}{the} \Ch{Am}{Almighty.}

	We \Ch{G}{forward} in this \Ch{Em}{generation}  \Ch{C}{triumphantly}.\Ch{D}{}
	\end{SBVerse}

	\begin{SBChorus}
	Won't you help to \Ch{G}{sing}

	\Ch{C}{these} \Ch{D}{songs} of \Ch{G}{freedom}?

	'Cause \Ch{C}{all I}    \Ch{D}{ever} \Ch{Em}{have}

	\Ch{C}{}    \Ch{D}{redemption} \Ch{G}{songs}. \Ch{C}{} \Ch{D}{}
 	\end{SBChorus}

	\begin{SBVerse}
	Emancipate \Ch{G}{yourselves} from mental \Ch{Em7}{slavery}

	none but \Ch{C}{ourselves} can \Ch{C/B}{free} our \Ch{Am}{minds}.

	Have no \Ch{G}{fear} for atomic \Ch{Em}{energy}

	'cause none of \Ch{C}{them} can \Ch{C/B}{stop} the \Ch{D}{time}.

	How long \Ch{G}{shall} they kill our \Ch{Em7}{prophets}

	While we \Ch{C}{stand} \Ch{C/B}{aside} and   \Ch{Am}{look}?

	Some say \Ch{G}{it's} just a part of \Ch{Em}{it}

	we've got \Ch{C}{to} \Ch{C/B}{fulfill} the    \Ch{D}{book.}
	\end{SBVerse}

	\begin{SBChorus}
	Won't you help to \Ch{G}{sing}

	\Ch{C}{these} \Ch{D}{songs} of \Ch{G}{freedom}?

	'Cause all \Chr{C}{I}    \Ch{D}{ever} \Ch{Em}{have}

	\Ch{C}{} \Ch{D}{Redemption} \Ch{G}{songs}.  \Ch{C}{} \Ch{D}{Redemption} \Ch{G}{songs.} \Ch{C}{}   \Ch{D}{Redemption} \Ch{G}{songs.} \Ch{C}{} \Ch{D}{}
	\end{SBChorus}

	\Ch{Em}{} \Ch{C}{} \Ch{D}{} x4

	\begin{SBVerse}
	REPEAT VERSE 2
	\end{SBVerse}

	\begin{SBChorus}
	Won't you help to \Ch{G}{sing}

	\Ch{C}{these} \Ch{D}{songs} of \Ch{G}{freedom}?

	'Cause all \Ch{C}{I}    \Ch{D}{ever} \Ch{Em}{have}

	\Ch{C}{}    \Ch{D}{Redemption} \Ch{G}{songs},    all \Ch{C}{I} \Ch{D}{ever} \Ch{Em}{have}

	\Ch{C}{}   \Ch{D}{Redemption} \Ch{Em}{songs}, \Ch{C}{these} \Ch{D}{songs} of \Ch{G}{freedom}

	\Ch{C}{}       \Ch{D}{songs} of \Ch{G}{freedom}
	\end{SBChorus}

	\chords{
		\chord{t}{x,p2,n,n,p3,p3}{C/B}
	}
\end{song}

\addcontentsline{toc}{section}{Errol Blackwood}
\begin{song}{Anyway the Wind Blows}{Gis}{1991}{Fred Mollin}{}{}
 
  \begin{SBChorus}
    \Ch{G#}{Anyway} the wind blows \Ch{C#}{}
    
    \Ch{D#}{it blows right back to} me
    
    little one

    Anyway the wind blows
    
    it blows right back to me
    
    listen to me
  \end{SBChorus}

	\begin{SBVerse}
	  \Ch{G#}{When} the morning comes
	  
	  \Ch{D#}{brings} the shining light
	  
	  \Ch{C#}{when} the night time comes
	  
	  \Ch{D#}{darkness} runs so bright
	\end{SBVerse}

  \begin{SBChorus}
    CHORUS
  \end{SBChorus}

	\begin{SBVerse}
	  When I let you go

	  said I set you free

	  if you come back to me

	  I will do it
	\end{SBVerse}

  \begin{SBChorus}
    CHORUS
  \end{SBChorus}

	\begin{SBVerse}
	  When the morning comes

	  brings the shining light

	  when the nighttime comes

	  darkness runs so bright
	\end{SBVerse}

  \begin{SBChorus}
    CHORUS
  \end{SBChorus}

\end{song}

\addcontentsline{toc}{section}{Sixto Rodriguez}
\begin{song}{Sugar Man}{A-moll}{1970}{Sixto Rodriguez}{}{}

\begin{SBVerse}
	Sugar \Ch{Am}{man}, \Ch{Am7+}{won't} you \Ch{Dm7}{hurry} \Ch{E7}{}

	'cos i'm \Ch{Dm7}{tired} of \Ch{E7}{these} \Ch{Am}{scenes}

	For a blue coin won't you bring back

	All those colors to my dreams.
\end{SBVerse}

\begin{SBChorus}
	\Ch{C}{Silver} \Ch{Am}{magic} \Ch{D7}{ships} you \Ch{F}{carry}

	\Ch{C}{Jumpers}, \Ch{Am}{coke}, sweet \Ch{F}{mary} \Ch{B}{jane} \Ch{E}{}
\end{SBChorus}

\begin{SBVerse}
	Sugar man met a false friend

	On a lonely dusty road

	Lost my heart when i found it

	It had turned to dead black coal
\end{SBVerse}

\begin{SBChorus}
	Silver magic ships you carry

	Jumpers, coke, sweet mary jane
\end{SBChorus}

\begin{SBVerse}
	Sugar man you're the answer

	That makes my questions disappear

	Sugar man 'cos i'm weary

	Of those double games l hear
\end{SBVerse}

	Sugar man...sugar man...sugar man...sugar man...sugar man...sugar man...sugar man...

\begin{SBVerse}
	Sugar man, won't you hurry

	'cos i'm tired of these scenes

	For a blue coin won't you bring back

	All those colors to my dreams.
\end{SBVerse}

\begin{SBChorus}
	Silver magic ships you carry

	Jumpers, coke, sweet mary jane
\end{SBChorus}

\begin{SBVerse}
	Sugar man met a false friend

	On a lonely dusty road

	Lost my heart when i found it

	It had turned to dead black coal
\end{SBVerse}

\begin{SBChorus}
	Silver magic ships you carry

	Jumpers, coke, sweet mary jane
\end{SBChorus}

\begin{SBVerse}
	Sugar man you're the answer

	That makes my questions disappear
\end{SBVerse}
\end{song}

\chapter{Po rosyjsku}
\begin{otherlanguage}{russian}
\addcontentsline{toc}{section}{Катюша}
\begin{song}{Катюша}{A-moll}{1938}{M. Isakovsky, M. Blanter}{}{}
	\begin{SBVerse}
	\Ch{Am}{Расцветали} \Ch{E7}{яблони} и груши,

	\Ch{E7}{Поплыли} \Ch{Am}{туманы} над рекой.

	\Chr{Am}{Вы}\Ch{F}{хо}\Ch{C}{дила} \Ch{Dm}{на} берег \Ch{Am}{Катюша},

	\Ch{Dm}{На} \Ch{Am}{высокий} \Ch{E7}{берег} на \Ch{Am}{крутой}.
	\end{SBVerse}
	\begin{SBVerse}
	Выходила, песню заводила

	Про степного, сизого орла,

	Про того, которого любила,

	Про того, чьи письма берегла.
	\end{SBVerse}
	\begin{SBVerse}
	Ой ты, песня, песенка девичья,

	Ты лети за ясным солнцем вслед.

	И бойцу на дальнем пограничье

	От Катюши передай привет.
	\end{SBVerse}
	\begin{SBVerse}
	Пусть он вспомнит девушку простую,

	Пусть услышит, как она поет,

	Пусть он землю бережет родную,

	А любовь Катюша сбережет.
	\end{SBVerse}
	\begin{SBVerse}
	Расцветали яблони и груши,

	Поплыли туманы над рекой.

	Выходила на берег Катюша,

	На высокий берег на крутой.
	\end{SBVerse}
\end{song}

\addcontentsline{toc}{section}{Ленинград}
\begin{song}{дикий мужчина}{H-moll}{2005}{Ленинград}{}{}

	\begin{SBVerse}
	\Ch{Hm}{Ты} называешь меня говнюком

	\Ch{Em}{Да}, я все время бухой

	\Ch{F#7}{И} твою жопу при людях хватаю

	\Ch{Hm}{Своей} волосатой рукой.
	\end{SBVerse}

	\begin{SBVerse}
	Да, мои ноги вонючие палки

	На которых все в дырках носки

	А эти две кучи из пыли и грязи -

	Это мои башмаки.
	\end{SBVerse}

	\begin{SBChorus}
	\Ch{Hm}{Да}, ты права, я - \Ch{F#7}{дикий} мужчина,

	Ты права, я - дикий мужчина,

	Ты права, я - дикий мужчина:

	Яйца, табак, перегар и щетина
	\end{SBChorus}

	\begin{SBChorus}
	Да, ты права, я - дикий мужчина,

	Ты права, я - дикий мужчина,

	Ты права, я - дикий мужчина:

	Яйца, табак, перегар и щетина
	\end{SBChorus}
\end{song}

\addcontentsline{toc}{section}{Кино}
\begin{song}{Группа Крови}{Fis-moll}{1988}{Кино}{}{}

\begin{SBVerse}
	\Ch{F#m}{Теплое} место

	Но улицы ждут \Ch{C#m}{oтпечатков} наших ног

	\Ch{F#m}{Звездная} пыль - \Ch{E}{на} сапогах.

	Мягкое кресло, клетчатый плед,

	Не нажатый вовремя курок.

	Солнечный день - в ослепительных снах.
\end{SBVerse}

\begin{SBChorus}
	\Ch{F#m}{Группа} крови - на рукаве,

	Мой порядковый \Ch{C#m}{номер} - на рукаве,

	\Ch{Hm}{Пожелай} мне удачи в бою, \Ch{E}{пожелай} мне:

	Не остаться в этой траве,

	Не остаться в этой траве.

	Пожелай мне удачи, пожелай мне удачи!
\end{SBChorus}

\begin{SBVerse}
	И есть чем платить, но я не хочу

	Победы любой ценой.

	Я никому не хочу ставить ногу на грудь.

	Я хотел бы остаться с тобой,

	Просто остаться с тобой,

	Но высокая в небе звезда зовет меня в путь.
\end{SBVerse}

\begin{SBChorus}
	Группа крови - на рукаве,

	Мой порядковый номер - на рукаве,

	Пожелай мне удачи в бою, пожелай мне:

	Не остаться в этой траве,

	Не остаться в этой траве.

	Пожелай мне удачи, пожелай мне удачи
\end{SBChorus}
\end{song}

\end{otherlanguage}
\end{document}

