\beginsong{Na Całość}[by={Pod Budą}]
  \beginverse
    \[G]{Piąta} \[D]{rano} za\[G]{bawa} skoń\[D]{czona},
    różo\[G\m]{wieje} już niebo na \[D]{wschodzie} \[A]{}
    Jakim słowem przywita mnie żona,
    bardziej święta niż proboszcz dobrodziej
    Głowa \[G]{ciężka} leciutkie kie\[D]{szenie} i w \[E\m]{łazience} unika się \[A]{lustra}
    bo naj\[D]{trudniej} z \[G]{obitym} su\[D]{mieniem}, \[G]{razem} z \[D]{sobą} \[A]{doczekać} do \[D]{jutra} \[A]{}
  \endverse
  \beginchorus
    A \[D]{jutro} \[A]{znów} pójdziemy na \[D]całość,
    za to wszystko co się dawno nie \[A]{udało}
    za dziew\[E\m]{czyny} które kiedyś nas nie \[A]{chciały}
    za ma\[E\m]{rzenia} które w chmurach się \[A]{rozwiały}
    za ko\[E\m]{legów} których jeszcze \[A]{paru} nam zo\[D]{stało} \[A]{}
  \endchorus
  \beginchorus
    A jutro znów pójdziemy na całość,
    miasto będzie patrzeć twarzą oniemiałą
    bo kto widział żeby z nocą się nie liczyć
    na dwa glosy nagle śpiewać no ulicy
    że w tym życiu to nam jakoś życia ciągle mało
  \endchorus
  \beginverse
    Tak mijają miesiące i kraje,
    coraz mądrzej gadają dokoła
    a my starym złączeni zwyczajem,
    nasze wojny toczymy przy stołach
    a nad ranem gdy boje skończone, wstaje słońce jak zwykle z ochotą
    w błogi spokój otulą nas żony i pozwolą zwyczajnie odpocząć
  \endverse
  \beginchorus
    Bo jutro znów idziemy na całość \dots
  \endchorus
\endsong
