\begin{song}{Na Całość}{}{}{Pod Budą}{}{}
  \begin{SBVerse}
    \Ch{G}{Piąta} \Ch{D}{rano} za\Ch{G}{bawa} skoń\Ch{D}{czona}, różo\Ch{Gm}{wieje} już niebo na \Ch{D}{wschodzie} \Ch{A}{}

    Jakim słowem przywita mnie żona, bardziej święta niż proboszcz dobrodziej

    Głowa \Ch{G}{ciężka} leciutkie kie\Ch{D}{szenie} i w \Ch{Em}{łazience} unika się \Ch{A}{lustra}

    bo naj\Ch{D}{trudniej} z \Ch{G}{obitym} su\Ch{D}{mieniem}, \Ch{G}{razem} z \Ch{D}{sobą} \Ch{A}{doczekać} do \Ch{D}{jutra} \Ch{A}{}
  \end{SBVerse}
  \begin{SBChorus}
    A \Ch{D}{jutro} \Ch{A}{znów} pójdziemy na \Ch{D}{całość, za to wszystko co się dawno nie} \Ch{A}{udało}

    za dziew\Ch{Em}{czyny} które kiedyś nas nie \Ch{A}{chciały}

    za ma\Ch{Em}{rzenia} które w chmurach się \Ch{A}{rozwiały}

    za ko\Ch{Em}{legów} których jeszcze \Ch{A}{paru} nam zo\Ch{D}{stało} \Ch{A}{}
  \end{SBChorus}
  \begin{SBChorus}
    A jutro znów pójdziemy na całość, miasto będzie patrzeć twarzą oniemiałą

    bo kto widział żeby z nocą się nie liczyć

    na dwa glosy nagle śpiewać no ulicy

    że w tym życiu to nam jakoś życia ciągle mało
  \end{SBChorus}
  \begin{SBVerse}
    Tak mijają miesiące i kraje, coraz mądrzej gadają dokoła

    a my starym złączeni zwyczajem, nasze wojny toczymy przy stołach

    a nad ranem gdy boje skończone, wstaje słońce jak zwykle z ochotą

    w błogi spokój otulą nas żony i pozwolą zwyczajnie odpocząć
  \end{SBVerse}
  \begin{SBChorus}
    Bo jutro znów idziemy na całość.....
  \end{SBChorus}
\end{song}
